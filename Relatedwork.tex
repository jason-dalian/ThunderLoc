
\section{Related Work}


%Based on the high position accuracy of the LMA (6�C12 m RMS in the horizontal and 20�C30 m RMS in the vertical for sources over the network) [ Thomas et al. , 2004 ], it can be used as
%reference (ground truth) to investigate the position accuracy of the acoustic array.
% Audible thunder (great than 20 Hz,3.3 to 500 Hz)
%: A major challenge for the early TOA systems was the need for precise time synchronization of multiple remote sensors.

%As a case study of acoustic source localization technology, 

% Lightning@SG, an App developed by National Environment Agency (NEA) that provides current lightning situation in Singapore, which uses lightning information from the latest lightning detection and location system to locate lightning. 
% It has four lightning detection sensors located in four cities.

%Thunder localization has drawn more attention for lightning location systems in recent years.
%According to the different frequency, 
Thunder localization system can use infrasonic signals, supersonic signals, and audible signals as the measurement data.
Due to space constraints, we can only mention a few related works using audible sound data.
%Audible thunder is thought to come from the expansion of the rapidly heated lightning channel.
Most of the existing thunder localization systems using audible sound data are based on centralized architecture.
Arechiga \emph{et al.} \cite{arechiga2011acoustic} studied acoustic reconstruction of lightning channel geometry using microphone arrays, improved upon thunder localization by using compact acoustic arrays with 1kS/s sample rate.
Akiyama \emph{et al.} \cite{akiyama1985channel} used rhombus array with 100k sample rate and 3.50m base line.
Few \emph{et al.} \cite{few1970lightning} implemented lightning channel reconstruction using 'Y' 4-element array with 1k sample rate and 30m base line.
MacGorman \emph{et al.} \cite{macgorman1978lightning} considered the effect of temperature and wind, presented a lightning localization system with square array with 0.5k sample rate and 50m base line.
Johnson \emph{et al.} \cite{johnson2011imaging}  used a network of broadband microphones, including a 4-element array with 1k sample rate, to locate the sources of thunder occurring during an electrical storm.
Qiu \emph{et al.} \cite{qiu2012synchronized}  used 3-element array with 100k sample rate and 12.5m base line to estimate the location of thunders.


In the past few years, there has been a growing interest for acoustic localization based on independent (unsynchronized) acoustic sensors, each made of two or more synchronized microphones. 
UbiK leveraged the dual-microphone on a mobile device to accurately localize the keystrokes \cite{UbiK}.
Zhu \emph{et al.} \cite{Zhu2014Context} proposed to utilize dual-microphone on three phones to identify keystrokes of a nearby keyboard based on
time difference of arrival (TDOA) measurements.
Liu \emph{et al.}  \cite{Liu2015Snooping}  achieved keystroke recognition by discriminating keystrokes based on TDOA of the keystroke sound at the dual-microphone of the off-the-shelf smartphone.
Wang \emph{et al.} \cite{wang2003acoustic} described a system having static cluster architecture, the system experienced a problem in that the accuracy decreased when an acoustic source occurred between the clusters.
Chen \emph{et al.} \cite{chen2004dynamic} showed that nodes in the system did not need to recognize their cluster head, reducing the constraints on deployment of the localization system.
Hu \emph{et al.}\cite{hu2009design} design the system based on 2-tier architecture, which experienced cost and deployment problems especially in the very large target area.
%Rabbat, \emph{et al.} \cite{rabbat2005robust} proposed a decentralized algorithm based on the distributed ML estimation technique using token ring architecture.
%Kim, \emph{et al.} \cite{kim2009locating} proposed to identify the node closest to the acoustic source, based on TOA comparisons between all nodes, thus incurring high communication cost and requiring global synchronization between all sensor nodes.
% Lightning is a method proposed in \cite{wang2008lightning} to identify the sensor closest to the acoustic source, also based on expensive broadcasting/flooding.
%Aarabi, \emph{et al.} ~\cite{aarabi1900fusion} used 10 dual-microphone arrays distributed in a room and used their data to locate three speakers.
%Wu, \emph{et al.} ~\cite{wu2012fusion} used three dual-microphone arrays to locate two sound sources in a distributed way in which only the local DOA estimates are communicated among arrays.
Canclini \emph{et al.}\cite{canclini2013acoustic} proposed a method for localizing an acoustic source with distributed microphone networks based on TDOA between microphones of the same sensor. 
%Most of the existing acoustic source localization methods in sensor networks are based on range-based measurement.
%In contrast, 
Different from these range-based methods, 
our proposed ThunderLoc is a range-free method, which can achieve robust localization performance.

The most related work to our ThunderLoc project is Lightning@SG, an App developed by National Environment Agency (NEA) in Singapore~\cite{LightningSG}. 
By utilizing the four sensors located in different cities, Lightning@SG monitors the lightning area based on current location of user, and pushes notifications of the lightning situation.
With the surging of smartphone sensing and wireless networking techniques,
crowdsensing has become a promising paradigm for cross-space and large-scale sensing. The iShake system uses smartphones as seismic sensors to measure and deliver ground motion intensity parameters produced by earthquakes~\cite{reilly2013mobile}.
Hu \emph{et al.} presented SmartRoad, a crowd-sourced road sensing system that detects and identifies traffic regulators, traffic lights, and stop signs~\cite{hu2013smartroad}.  
Zhou \emph{et al.} presented  a novel bus arrival time prediction system based on crowd-participatory sensing~\cite{zhou2012long}. 
Motivated by these works, we uses dual-microphone smartphones as sensors to estimate the position of the thunders by crowdsensing mechanisms.
