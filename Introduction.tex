
\section{Introduction}

%\cite{guo2011localising}
%\cite{sallai2011acoustic}
%\cite{allen2008voxnet}

%Thunder is the sound made by lightning.  Thunder is seldom heard at distances over 20 km. Thunder can be treated as a weighted superposition of different channels�� acoustic signatures.
%The distance of the lightning can be calculated by the listener depending on how long the sound is heard since the vision of the lightning strikes.
%The sudden increase in pressure and temperature from lightning produces rapid expansion of the air surrounding and within a bolt of lightning.



Lightning in the sky is a direct threat to air and ground-based operations, and it is also a reflection of other destructive forces associated with thunderstorms. 
Thunder is the acoustic shock wave resulting from a sudden and intense heating of the air in the lightning channel~\cite{few1995acoustic}.
Thunder localization system has a variety of applications, such as lightning channel reconstruction, risk analysis of faults in power systems, 
and electromagnetic interference studies~\cite{cummins2009overview}. 
Lightning location systems (LLS) represent a promising source of experimental data to be used for the development of standards related to the protection of power systems against lightning.
Cloud-to-ground lightning is the single largest cause of transients, faults, and outages in electric power transmission and distribution systems in lightning-prone areas. 
Additionally, lightning is a major cause of electromagnetic interference that can affect all electronic systems. 
Depending on the nature of the lightning and distance of the listener, thunder can range from a sharp, loud crack to a long, low rumble~\cite{dayeh2015first}.
In this paper, we focus on the location of the loud clap, or crack of thunder. 


In the last decade, electromagnetic radiation-based lightning location systems need the expensive special device~\cite{vahabi2015modeling}.
In recent years, the centralized microphone array-based solution to thunder localization exploited multiple synchronized microphones to simultaneously acquire acoustic signals, 
which have some limitations with regard to the distances between the microphones, and sensing range for the large-scale applications. %~\cite{arechiga2011acoustic}.
Wireless acoustic sensor networks (WASNs) can overcome these limitations. 
A WASN consists of a set of wireless microphone nodes that are spatially distributed over the environment, usually in an ad-hoc fashion.
Due to the wireless communication, the array-size limitations disappear and WASN can cover much larger areas.
Acoustic source localization in sensor networks has been widely studied in the recent decade.
Generally, each node just has a single microphone element in previous distributed acoustic source localization system.
In the past few years, there has been a growing interest for acoustic nodes made of two or more synchronized microphones.
Aarabi \emph{et al.} used 10 dual-microphone arrays distributed in a room and used their data to locate three speakers~\cite{aarabi1900fusion}.
Wu \emph{et al.} used three dual-microphone arrays to locate two sound sources in which only the local DOA estimates are communicated among arrays~\cite{wu2012fusion}. 
Most of the systems need to design the special hardware to capture the multiple channel synchronized audio signal. 
With the recent advances in mobile computing and communication technology, most mobile devices (e.g. smart phones, tablets and laptops) are equipped with dual-microphone onboard.
Thunder localization in WASN with dual-microphone nodes is becoming feasible due to recent advances in personal portable computing devices with the rapid deployment ability.


% The demand for thunder localization comes not only from the scientific community, but also from public and private groups. 
% WWLLN currently have over 70 sensors around the globe to detect the lightning. 
% WWLLN detects very low frequency (VLF) radio waves (3-30 kHz) emitted by a lightning strike, then the central processors calculate lightning locations using the continuous flow of wave from WWLLN sensors. 
% Earth Networks Inc. offers a service called Earth Networks Total Lightning Network (ENTLN) that provides in-cloud and cloud-to-ground lightning detection. 
% It uses time-of-arrival detection methodology with GPS technology to accurately locate lightning. 
% It includes more than 1,200 sensors in more than 40 countries around the world. 
% Lightning mapping array (LMA) is a network of time-of-arrival sensors that passively receives very high frequency (VHF) impulses from electrical breakdown within thunderstorms. 
% The North Alabama Lightning Mapping Array (NALMA) is a three dimensional very high frequency detection network of 11 VHF receivers. 


In this paper, we propose a crowdsensing mechanism for thunder localization by leveraging dual-microphone smartphones.
With commodity dual-microphone smartphones, the binary left/right data from dual-microphone of each smartphone is effectively
collected and utilized to estimate the location of the thunder.
The proposed system solely relies on the collaborative effort of the participating users to achieve thunder localization.
The ThunderLoc project created such a prototype system with different types of Android based mobile phones, and validated our approach with a virtual thunder study, showcasing the potential of the proposed solution. 
ThunderLoc takes advantage of the sensing capabilities, including the acoustics, position and orientation offered by the microphone, GPS, accelerometer and magnetometer on the smartphones.
The key idea of ThunderLoc is the division of a 2D localization space into distinct regions by the perpendicular bisectors of lines joining dual-microphone in each smartphone. 
Each distinct region formed in this manner can be uniquely identified by a binary sequence.
ThunderLoc constructs the binary sequence table that maps all these feasible binary sequences to the corresponding regions by using the positions and orientations of the smartphone nodes.
The smartphone nodes determine the measured binary sequence based on the sign of TDOA between dual-channel acoustic signals of each smartphone node.
The location of the thunder is estimated by searching through the binary sequence table to determine the nearest feasible sequence to the measured sequence. 
%To our knowledge, thunder localization with smartphone-based crowdsensing has not been considered in the literature before.
Our contributions are as follows:

\begin{itemize}
\item{We designed, implemented, deployed and  evaluted a crowdsensing-based
thunder localization system called ThunderLoc. Our ThunderLoc app for Android platform has
been downloaded by more than 1,000 users in our university to monitor the thunder near our campus. 
To the best of our knowledge, ThunderLoc is the first crowdsensing-based thunder localization service;}
\item{High-accuracy time synchronization among smartphones is not required. ThunderLoc rely on measuring the sign of TDOA between two synchronized microphones of each smartphone, 
and the requirement of time synchronization is reduced;}
\item{Error-tolerant localization scheme. The binary left/right data and novel localization method make the localization system more robust to the location error, direction error and  measurement error of smartphone nodes;}
\item{ Low communication overhead and computational complexity. 1-bit measurement information is passed by each smartphone in the sensor network, and the simple cross correlation algorithm is good enough to estimate the binary left/right data.}
\end{itemize}


The rest of the article is organized as follows. Section \uppercase\expandafter{\romannumeral 2} presents an overview of the ThunderLoc system.
Then, the design of ThunderLoc is introduced in section \uppercase\expandafter{\romannumeral 3}.
Section \uppercase\expandafter{\romannumeral 4} discusses several issues concerning practical system deployment.
Section \uppercase\expandafter{\romannumeral 5} presents simulation results and an empirical evaluation on
the test-bed. Section \uppercase\expandafter{\romannumeral 6} briefly surveys related work.
Section \uppercase\expandafter{\romannumeral 7} concludes the whole article.






