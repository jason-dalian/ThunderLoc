
\section{Conclusions and Future work}

In this paper, we have designed ThunderLoc, a novel thunder localization system that leverages dual-microphone smartphones to achieve thunder localization using unreliable binary node sequences by crowdsensing mechanism.
The proposed design formulates the thunder localization as the searching problem in Hamming space by making use of the binary sequence from the dual-microphone smartphones.
Since our ThunderLoc system runs on COTS smartphones and supports spontaneous setup, it has potential to enable a wide range of thunder localization systems.
Besides the basic design, a robust localization method is proposed to enhance the localization performance in the practical application.
Our system is verified and evaluated through analysis, extensive simulation as well as the test-bed experimentation.
The test results have shown that the proposed method can effectively implement thunder localization with crowdsensing mechanism.
Our next step is to study the motivate mechanism to increase the participation rate of crowdsensing. 
Incremental implement of ThunderLoc system in the cloud computing platform is another part of our future work.

%%%12345
% \section{Acknowledge}

% This work is supported by Natural Science Foundation of China (Grants No. 61272524 and No.61202443) and the Fundamental Research Funds for the Central Universities (Grants No.DUT15QY05 and No.DUT15QY51).
% This work is also supported by Specialized Research Fund for the Doctoral Program of Higher Education (Grant No. 20120041120049).
